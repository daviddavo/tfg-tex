\section{Encaminamiento}

\subsection{Switching}
\begin{frame}[allowframebreaks]{Tipo de Switching}
    \begin{itemize}
        \item \textbf{Store-and-forward}
        \begin{itemize}
            \item No proporciona a penas ventajas en nuestro contexto
            \item Hay que almacenar los paquetes enteros, por lo que necesita demasiada área
        \end{itemize}
        \item \textbf{Cut-through}
        \begin{itemize}
            \item Sigue necesitando búfers para almacenar el paquete cuando se produce una contención
            \item Menos latencia
            \item Degrada a store-and-forward en routing no adaptativo
        \end{itemize}
        \item \textbf{Wormhole}
        \begin{itemize}
            \item Se parten los paquetes en \textit{flits}. Se manda un flit cabecera que reserva el camino, y uno cola que lo libera. Solo hay que guardar el flit (no todo el paquete) en caso de contención.
            \item Se puede mitigar la contención (y evitar deadlocks) con canales virtuales
        \end{itemize}
        \item \textbf{Conmutación de circuitos}
        \begin{itemize}
            \item Se establece un circuito antes de comenzar la transmisión (más latencia que wormhole)
            \item El circuito está reservado, por lo que se aprovecha el ancho de banda al máximo en esa conexión (pero no lo pueden usar otras conexiones)
        \end{itemize}
    \end{itemize}
    \begin{exampleblock}{Conclusión}
        Usamos wormhole (o conmutación de circuitos si vemos que no funciona)
    \end{exampleblock}
\end{frame}

\subsection{Determinista o adaptativo}
\begin{frame}{¿Determinista o adaptativo?}
    \begin{columns}
        \begin{column}{.5\textwidth}
            \begin{block}{Determinista}
                \begin{itemize}
                    \item Fácil de implementar (Ej: DOR) % \footnote{DOR: Dimensional Order Routing})
                    \item No hay \textbf{livelocks}
                    \item Para solucionar el \textbf{starvation} podemos usar RR
                    \item Pueden evitarse deadlocks con algún esquema concreto
                \end{itemize}
            \end{block}
        \end{column} %
        \begin{column}{.5\textwidth}
            \begin{block}{Adaptativo}
                \begin{itemize}
                    \item Necesitamos almacenar de alguna manera el estado de la red $\rightarrow$ buffers $\rightarrow$ más área
                    \item Necesitamos comunicar el estado de la red $\rightarrow$ gossiping protocols
                    \item Hay que tratar los \textbf{livelocks}
                \end{itemize}
            \end{block}
        \end{column}%
    \end{columns}
    
    \begin{exampleblock}{Conclusión}
        Usar un algoritmo determinista, aunque puede merecer la pena buscar un algoritmo adaptativo sencillo
    \end{exampleblock}
\end{frame}

\subsection{Desde la fuente o desde los nodos}
\begin{frame}{¿Encaminamiento en la fuente o en los nodos?}
    \begin{columns}
        \begin{column}{.5\textwidth}
            \begin{block}{En la fuente}
                \begin{itemize}
                    \item La interfaz de cada elemento necesita conocer la red
                    \item (Y su estado en el caso de encaminamiento adaptativo)
                    \item Buena idea si quieres adaptativo con rúters ligeros (pero NIC grandes)
                \end{itemize}
            \end{block}
        \end{column}%
        \begin{column}{.5\textwidth}
            \begin{block}{En cada nodos}
                \begin{itemize}
                    \item Cada rúter necesita conocer la red
                    \item (Y su estado en el caso de encaminamiento adaptativo)
                    \item Con encaminamiento determinista puede calcularse el siguiente salto con una operación aritmética combinacional o una pequeña ROM
                \end{itemize}    
            \end{block}
        \end{column}
    \end{columns}
    \begin{exampleblock}{Conclusión}
        Encaminamiento en los nodos
    \end{exampleblock}
\end{frame}