\section{Características de la red}

\subsection{Topología}
\begin{frame}{Topología}
    \begin{itemize}
        \item Toro 2D y cada nodo tiene 5 enlaces: NSEO+\textit{local}
        \item Malla/Toro 2D y los nodos tienen 4 enlaces
        \begin{itemize}
            \item En malla, los de los márgenes conectan con dispositivos
            \item En toro, algunos conectan con dispositivos y tres direcciones, y otros con NSEO (sin disp)
        \end{itemize}
    \end{itemize}
    \begin{exampleblock}{Conclusión}
        TBD
    \end{exampleblock}
\end{frame}

\subsection{Dentro de los nodos}
\begin{frame}{Contención en los nodos}
    \begin{itemize}
        \item \textbf{No bloqueante} $\rightarrow$ crossbar 4x4 o 5x5
        \item \textbf{Bloqueante} $\rightarrow$ red MIN u otra cosa (menos transistores)
    \end{itemize}
    
    \begin{alertblock}{¿Y el arbitraje?}
        Si dos o más entradas distintas quieren ir a la misma salida, podemos usar round robin.
    \end{alertblock}
\end{frame}