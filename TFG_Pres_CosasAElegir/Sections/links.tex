\section{Enlaces}

\subsection{Sentido de transmisión}
\begin{frame}{Sentido de transmisión}
    \begin{columns}
        \begin{column}{.5\textwidth}
            \begin{alertblock}{Half-duplex}
                \begin{itemize}
                    \item \textbf{Ventaja}: Usamos la mitad de cable
                    \item Necesario crear algún tipo de arbitraje (más área gastada)
                    \item Pueden producirse colisiones
                    \item Mayor latencia y menor throughput en general
                    \item El objetivo de usar una NoC es quitarnos el medio compartido
                \end{itemize}
            \end{alertblock}
        \end{column} %
        \begin{column}{.5\textwidth}
            \begin{block}{Full-duplex}
                \begin{itemize}
                    \item En el silicio, los cables son lo más barato y fácil de implementar
                    \item Al no ser necesario arbitraje, nos ahorramos área (y tiempo diseñando)
                \end{itemize}
            \end{block}
        \end{column} %
    \end{columns}
    
    \begin{exampleblock}{Conclusión}
        Usaremos full-duplex
    \end{exampleblock}
\end{frame}

\subsection{Segmentación}
\begin{frame}{Segmentación}
    \begin{itemize}
        \item Podríamos segmentar los enlaces para aumentar el throughput reduciendo el tiempo de ciclo
    \end{itemize}

    \begin{exampleblock}{Conclusión}
        Dependiendo del ciclo de reloj objetivo y la latencia de los enlaces implementados, se segmentarán o no los enlaces.
    \end{exampleblock}
\end{frame}