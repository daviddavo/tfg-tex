\begin{otherlanguage}{english}
\chaptertype{E}
\addchap{Abstract}

Network on a Chip (NoC) is an emergent technology alternative to traditional interconnection methods, in which network design methods are applied to the connexions between modules. They allow greater scalability, abstraction, flexibility and resilience than conventional on-chip networks.

Although NoCs have been widely used to connect processors and complex devices, in this Final Degree Project, we apply this technology inside the execution unit of the RISC-V based processor SWerv-EL2.

Secondly, using a dynamically partially reconfigurable FPGA permits adding, moving, replacing and removing heterogeneous elements at run-time, enabling us to insert features as needed or to provide redundancy by replacing defective modules to make our design fault-tolerant.

For this, we have studied in-depth and designed a NoC, trying to minimize the resources utilized and its impact on the core. Afterwards, we modified the execution unit of the core to include said network, using it to interconnect its submodules. In both cases, we used the hardware description language SystemVerilog.

In the end, we discuss the problems encountered during the project, the results and conclusions, and possible future works.

\vfill
\begin{keywords}[title=Keywords]
% \small RISC-V, Red en Chip, NoC, Sistemas en Chip, Arquitectura de Computadores, Reconfiguración Parcial Dinámica, FPGA, SystemVerilog
\small RISC-V, Network on Chip, NoC, System on Chip, Computer Architecture, Dynamic Partial Reconfiguration, FPGA, SystemVerilog
\end{keywords}
\end{otherlanguage}