\addchap{Resumen}

Las redes en chip (NoC) son una tecnología emergente alternativa a los métodos de interconexión convencionales en la que se aplican los métodos de diseño de redes a las conexiones entre módulos. Permiten una mayor escalabilidad, abstracción, flexibilidad y resiliencia que las conexiones intra-chip convencionales.

Aunque las NoC han sido ampliamente usadas para la intra-comunicación entre procesadores y dispositivos complejos, en este Trabajo de Fin de Grado aplicamos esta metodología dentro de la unidad de ejecución de un procesador basado en la arquitectura libre RISC-V: el SWerv-EL2.

Por otro lado, en las FPGAs la reconfiguración dinámica posibilita añadir, sustituir y eliminar elementos heterogéneos en tiempo de ejecución, permitiéndonos incorporar funcionalidades conforme van siendo necesarias, o aportar redundancia y reemplazar módulos defectuosos para hacer nuestro diseño más tolerante a fallos.

Para ello, hemos estudiado a fondo y diseñado una NoC, tratando de minimizar los recursos consumidos y el impacto en área de la misma en el procesador. Posteriormente modificamos la unidad de ejecución del procesador para incluir dicha red como interconexión entre los módulos de dicha unidad. En ambos casos usaremos el lenguaje de descripción de hardware SystemVerilog.

Finalmente se comentan los problemas encontrados durante el proyecto, los resultados y conclusiones, y el trabajo futuro posible para la continuación de este proyecto.

\vfill
\begin{keywords}[title=Palabras clave]
\small RISC-V, Red en Chip, NoC, Sistemas en Chip, Arquitectura de Computadores, Reconfiguración Parcial Dinámica, FPGA, SystemVerilog
\end{keywords}