\setchapterpreamble[u]{
    \dictum[Leonard Nimoy como Mr. Spock en \textit{Start Trek TOS}]{\textquote{Computers make excellent and efficient servants, but I have no wish to serve under them.}}
}
\chapter{RISC-V}

% RISC-V es una arquitectura de estándar abierto que sigue los principios de las arquitecturas RISC (\textit{Reduced Instruction Set Computers} o Computadoras de Conjunto Reducido de Instrucciones), de tipo load-store y conjunto de instrucciones variable y modular, y 32 registros de propósito general% (el registro cero está conectado a tierra)\footnote{También existe un subconjunto para procesadores embebidos de 16 registros}.

% RISC-V es tan solo un ISA (Instruction Set Architecture), es decir, una definición de un conjunto de instrucciones que debe soportar cualquier procesador que quiera ejecutar software compatible con dicha arquitectura.
RISC-V es tan solo un ISA (\textit{Instrucion Set Architecture}), es decir, la parte del procesador que debe tener en cuenta el compilador o el programador para generar software ejecutable por una máquina de esa arquitectura \cite{HenessyPattersonCAQA}. La principal diferencia con otras ISAs como MIPS, ARM o x86 es que RISC-V es una arquitectura de estándar abierto y gratuito, por lo que cualquiera puede diseñar su propio procesador que la implemente. 
% A diferencia de AMD, Intel o ARM, \mbox{RISC-V} International no fabrica ningún tipo de procesador; en su lugar diversas compañías y organizaciones crean sus propias implementaciones \cite{RiscVExchangeCores}, con distintas características. 

La arquitectura RISC-V tiene un conjunto de instrucciones \textit{reducido} o RISC (\textit{Reduced Instruction Set Computer}), por lo que cuenta con menos instrucciones que una arquitectura compleja o CISC (\textit{Complex Instruction Set Computer}). Al contrario que en CISC, las instrucciones en RISC son más simples y cada una de ellas realiza una sola función muy concreta. Asimismo, decimos que RISC-V es una arquitectura \textit{load/store}, también llamada de \textit{registro a registro}, en la que el procesador cuenta con un \textit{banco de registros} en el que se realizan las operaciones, y nunca directamente sobre la memoria. Es decir, para ejecutar un cálculo usando datos de la memoria primero es necesario usar instrucciones de \textit{load} para llevar los operandos a los registros, luego realizar el cálculo con todas las instrucciones aritmético-lógicas necesarias y, finalmente, usar un \textit{store} para desplazar el resultado de nuevo a la memoria.

 La arquitectura RISC-V es modular, existiendo distintos conjuntos base, y extensiones que pueden añadir funcionalidades a un procesador. Para que un procesador soporte la arquitectura RISC-V no privilegiada debe implementar el conjunto base \textbf{I} de 32, 64 o 128 bits; aunque puede usarse el conjunto base \textbf{E} de 32 bits para sistemas embebidos, que cuenta con la mitad de registros accesibles \cite{RiscVSpec1}. Asimismo, se definen múltiples extensiones que pueden soportarse en cada implementación. A continuación se listan las extensiones no privilegiadas ratificadas en la versión 20191213 del ISA:

\begin{itemize}[noitemsep]
    \item[\textbf{M}] Multiplicación y división entera
    \item[\textbf{A}] Instrucciones atómicas
    \item[\textbf{F}] Coma flotante de precisión sencilla (IEEE-754)
    \item[\textbf{D}] Coma flotante de precisión doble (IEEE-754)
    \item[\textbf{Q}] Coma flotante de precisión cuádruple
    \item[\textbf{C}] Instrucciones comprimidas (como el conjunto \textit{Thumb} de ARM)
    \item[\textbf{Zicsr}] CSR: Registro de Control y Estado
    \item[\textbf{Zifencei}] Barrera o \textit{fence} para la fase \textit{fetch} de instrucciones
\end{itemize}

También cuenta con extensiones aún no ratificadas para manipulación de bits, SIMD (\textit{Single Instruction Multiple Data}), operaciones vectoriales, criptografía, JIT (aceleración de lenguajes intepretados), virtualización... Parte del espacio de instrucciones está reservado para la implementación de conjuntos de instrucciones propietarios.

% \begin{recuadro}
Para identificar las características de la arquitectura de un procesador, suele usarse como nomenclatura el prefijo \textit{RV}, seguido del número de bits y los subconjuntos de instrucciones que soporta (comenzando con la ISA base). Por ejemplo, un procesador de propósito general de 64 bits con multiplicación y división de enteros, instrucciones atómicas y coma flotante sería un procesador RV64IMAFD; mientras que un procesador embebido con la extensión criptográfica sería un RV32EK.

En ocasiones nos referiremos a la extensión \textbf{G}, que es una abreviatura de \textbf{IMAFDZicsrZifencei} y comprende todos los conjuntos necesarios para implementar un procesador de propósito general. Por lo tanto, un procesador RV32/64G es un objetivo estable para el desarrollo de software, pues todas sus extensiones están ratificadas.
% \end{recuadro}

% En caso de querer implementar un dispositivo embebido, es posible usar como conjunto base el subconjunto \textbf{E} en lugar de \textbf{I}, que solo tiene acceso a 16 de los 32 registros entre otras muchas limitaciones.

% \section{Formato de las instrucciones}

El formato de las instrucciones es el siguiente:

\begin{figure}[h]
\begin{center}
\setlength{\tabcolsep}{4pt}
\begin{tabular}{p{0.3in}@{}p{0.8in}@{}p{0.6in}@{}p{0.18in}@{}p{0.7in}@{}p{0.6in}@{}p{0.6in}@{}p{0.3in}@{}p{0.5in}l}
\\
\multicolumn{1}{c}{\instbit{31}} &
\instbitrange{30}{25} &
\instbitrange{24}{21} &
\multicolumn{1}{c}{\instbit{20}} &
\instbitrange{19}{15} &
\instbitrange{14}{12} &
\instbitrange{11}{8} &
\multicolumn{1}{c}{\instbit{7}} &
\instbitrange{6}{0} \\
\cline{1-9}
\multicolumn{2}{|c|}{funct7} &
\multicolumn{2}{c|}{rs2} &
\multicolumn{1}{c|}{rs1} &
\multicolumn{1}{c|}{funct3} &
\multicolumn{2}{c|}{rd} &
\multicolumn{1}{c|}{opcode} &
R-type \\
\cline{1-9}
\\
\cline{1-9}
\multicolumn{4}{|c|}{imm[11:0]} &
\multicolumn{1}{c|}{rs1} &
\multicolumn{1}{c|}{funct3} &
\multicolumn{2}{c|}{rd} &
\multicolumn{1}{c|}{opcode} &
I-type \\
\cline{1-9}
\\
\cline{1-9}
\multicolumn{2}{|c|}{imm[11:5]} &
\multicolumn{2}{c|}{rs2} &
\multicolumn{1}{c|}{rs1} &
\multicolumn{1}{c|}{funct3} &
\multicolumn{2}{c|}{imm[4:0]} &
\multicolumn{1}{c|}{opcode} &
S-type \\
\cline{1-9}
\\
\cline{1-9}
\multicolumn{1}{|c|}{imm[12]} &
\multicolumn{1}{c|}{imm[10:5]} &
\multicolumn{2}{c|}{rs2} &
\multicolumn{1}{c|}{rs1} &
\multicolumn{1}{c|}{funct3} &
\multicolumn{1}{c|}{imm[4:1]} &
\multicolumn{1}{c|}{imm[11]} &
\multicolumn{1}{c|}{opcode} &
B-type \\
\cline{1-9}
\\
\cline{1-9}
\multicolumn{6}{|c|}{imm[31:12]} &
\multicolumn{2}{c|}{rd} &
\multicolumn{1}{c|}{opcode} &
U-type \\
\cline{1-9}
\\
\cline{1-9}
\multicolumn{1}{|c|}{imm[20]} &
\multicolumn{2}{c|}{imm[10:1]} &
\multicolumn{1}{c|}{imm[11]} &
\multicolumn{2}{c|}{imm[19:12]} &
\multicolumn{2}{c|}{rd} &
\multicolumn{1}{c|}{opcode} &
J-type \\
\cline{1-9}
\end{tabular}
\end{center}
\caption[Formato de instrucciones base en RISC-V 32 bits]{Formato de instrucciones base de 32 bits. Cada subcampo del inmmediato se describe con la posición de los bits (imm[{\em x}\,]) relativa al subcampo, y no a la instrucción como suele hacerse. Extraído de \cite{RiscVSpec1} }
\label{fig:baseinstformats}
\end{figure}

% En el capítulo \ref{ch:desCore} de la página \pageref{ch:desCore} se exponen distintas implementaciones RV32/64 y la elegida para incluir el diseño de la NoC.

La codificación de las instrucciones es fija, y podemos distinguir los siguientes seis formatos en el conjunto base~\cite{BerkeleyCS61C7}:

\begin{itemize}[noitemsep]
    \item [\textbf{R}] Instrucciones en las que se especifican 3 registros: 2 que almacenan los operandos fuente y un destino en el que guardar el resultado. Para poder realizar más operaciones, este tipo de instrucciones cuentan con bits extra para seleccionar la operación a realizar.
    \item [\textbf{I}] Instrucciones que incluyen un operando \textit{inmediato} (fijo en la codificación de la instrucción), además de otro registro fuente y un destino.
    \item [\textbf{S}] Instrucciones de \textit{store}, encargadas de guardar los datos de un registro a la memoria.
    \item [\textbf{SB}] Instrucciones de ramificación \textit{branch}, que implementan saltos condicionales.
    \item [\textbf{U}] Instrucciones en las que se especifica un registro destino y un inmediato (de 20 bits, más largo que en las de tipo I).
    \item [\textbf{UJ}] Instrucciones de salto incondicional.
\end{itemize}

\section{Elección del núcleo RISC-V}
\label{sec:desCore}

Para seleccionar el núcleo (\textit{core}) a modificar, se ha explorado la lista de \textit{cores} de RISC-V Exchange \cite{RiscVExchangeCores}. El objetivo principal ha sido buscar un proyecto en el que fuese sencillo embeber una NoC como prueba de concepto, sin importar mucho las características técnicas del procesador, su potencia o sus resultados en \textit{benchmarks}. Por lo tanto, hemos priorizado que su lenguaje de programación fuese conocido, tuviese licencias abiertas, y una amplia documentación y herramientas de desarrollo.

Se ha filtrado la selección buscando Cores escritos en lenguajes conocidos (VHDL o SystemVerilog) y con licencia libre reconocida por OSI (Apache, BSD, GPL o MIT) \cite{OSILicenses}. Finalmente, la selección se redujo a los siguientes núcleos:

\begin{description}
    \item[SweRV Core EH1] [RV32IMCZ]: Es superescalar con 2 vías de lanzamiento (dual issue) y segmentación en 9 fases. Ha sido fabricado en tecnología de 28nm, pero cuenta con optimizaciones para FPGA. Creado por Western Digital. Tiene 4 unidades aritmético-lógicas (ALU), 2 \textit{early} y 2 \textit{late}, cada una conectada a una de las vías del ILP \cite{RepoSwervEH1}.
    \item[SweRV Core EH2] [RV32IMACZ]: Basado en el EH1, se le añade multithreading para 2 hilos, y por lo tanto también se le agregan instrucciones atómicas \cite{RepoSwervEH2}.
    \item [SweRV Core EL2] [RV32IMC]: Es un núcleo mucho más sencillo que el EH1 creado para reemplazar a las máquinas de estados y otras funciones lógicas implementadas en Sistema en Chip (haciendo que puedan ser actualizadas). La segmentación es en solo 4 etapas y es escalar \cite{RepoSwervEL2}.
    \item [biRISC-V] [RV32IMZicsr]: Superescalar con dual issue y segmentación en 6-7 etapas, tiene un divisor hardware, 1 unidad de load-store y 2 ALUs. Implementa también instrucciones privilegiadas, por lo que puede ejecutar sistemas operativos de propósito general \cite{RepoBiRISCV}.
    \item [PicoRV32] [RV32IMC]: Es un pequeño procesador de menos de 2000 LUTs, por lo que caben múltiples en una FPGA. % Una idea alternativa hubiese sido crear una interfaz de red para este procesador, y conectar varios de ellos en una FPGA formando un SoC manycore \cite{PicoRV32gh}.
\end{description}

Algunos otros núcleos y herramientas que se consideraron y fueron descartados en fases tempranas (principalmente por el lenguaje de programación o su complejidad) fueron los siguientes:

\begin{description}
    \item [WARP-V] Está escrito en TL-Verilog y es un generador de cores con muchas opciones de configuración. Parte de este trabajo podría haber sido añadir al WARP-V la opción de usar una NoC, al igual que existen opciones para añadir o quitar etapas del pipeline.
    \item [Rocket Chip] Más que un procesador, se trata de un generador escrito en \textbf{Chisel}, lenguaje libre de definición de hardware basado en Scala que emite código sintetizable en Verilog. Ha sido diseñado por el departamento de D. Patterson, A. Waterman y K. Asanović, creadores del RISC-V. \cite{AsanovicRocketChip}.
    \item [BOOM] [RV64GC] \textit{Berkeley Out-of-Order Machine}. Está creado por el mismo departamento que Rocket y con herramientas muy similares. Su característica principal es su alto grado de paralelismo a nivel de instrucción. Sin embargo, su estructura es mucho más compleja y, por lo tanto, difícil de modificar \cite{Celio:EECS-2015-167}.
\end{description}

La principal ventaja de Rocket y BOOM es que, debido a su longevidad y soporte a lo largo del tiempo, cuentan con amplia documentación, literatura, herramientas, y han sido implementados tanto en FPGA como en ASICs, por lo que aunque no han sido usados directamente en este proyecto, sí que se ha consultado en ocasiones literatura relacionada con estos procesadores.

\begin{figure}[h]
    \centering
    \includegraphics[width=\linewidth]{images/external/roadmap_compare.png}
    \caption[Comparativa de la potencia de varios Cores RISC-V de código abierto]{Comparativa de la potencia de varios Cores RISC-V de código abierto. Resaltado en rojo los procesadores de la familia SweRV. Extraído de \citetitle{SweRVRoadmap}~\cite{SweRVRoadmap}.}
    \label{fig:swerv_comparative}
\end{figure}

Finalmente, se eligió la familia SweRV debido a la familiaridad con el proyecto (se usa este procesador en la actividad formativa ``Implementación ASIC de 28nm del procesador RISC V"). En la figura \ref{fig:swerv_comparative} se muestra una comparativa de dichos procesadores con otros cores de código abierto. Dentro de los procesadores de esta familia, se decidió modificar el EL2 por su sencillez y los pocos recursos humanos de los que se dispone en este proyecto. No obstante, tratándose de una prueba de concepto, siempre es posible en el futuro aplicar las técnicas aprendidas a otros productos más complejos, con mejoras y optimizaciones.

\section{El procesador SweRV-EL2}

El procesador SweRV-EL2 ha sido creado por Western Digital y publicado por la colaboración de código abierto CHIPS Alliance\footnote{CHIPS: \textit{Common Hardware for Interfaces, Processors and Systems}} en 2020. Es la segunda generación de la familia SweRV de procesadores, siendo de estos el más ligero y menos potente. Ha sido fabricado a 16 nm en TSMC con un área total de $0.023mm^2$, y funcionando con una frecuencia objetivo de 600MHz. Su repertorio de instrucciones es el RV32IMC+Zbb+Zbs (Zbb y Zbs son instrucciones de manipulación de bits aún no ratificadas), por lo que pueden realizarse multiplicaciones y divisiones hardware.

Además del core, el procesador cuenta con otros elementos periféricos: caché de Instrucciones, \textit{Closely-Coupled Memory} (CCM) (tanto de datos, DCCM, como instrucciones, ICCM), interfaces de debug y AXI de 64 bits...

Su descripción de alto nivel está parametrizada con multitud de opciones de configuración que podemos fijar manualmente o ejecutando la herramienta \textit{swerv\_config\_gen}, por ejemplo: si queremos optimizar para FPGA, si queremos que incluya puerto AXI, el tamaño de las cachés y sus políticas de funcionamiento, la configuración del predictor de salto... La herramienta también cuenta con cuatro perfiles objetivo predefinidos:
\begin{itemize}[noitemsep]
    \item \textbf{default}: Configuración por defecto con una interfaz bus AXI4.
    \item \textbf{default\_ahb}: Configuración por defecto con una interfaz bus AHB.
    \item \textbf{typical\_pd}: Se le quita la ICCM y tiene una interfaz bus AXI4. Es la usada para la fabricación en ASIC y la única que tiene desactivadas las optimizaciones para FPGA.
    \item \textbf{high\_perf}: Configuración para alto rendimiento con interfaz bus AXI4. Para lograr un mejor rendimiento se aumentan algunos recursos, como el tamaño del \textit{Branch Target Buffer} o el \textit{Branch History Buffer}, logrando mejores predicciones de salto.
\end{itemize}

En este proyecto hemos elegido la configuración \textit{default} con optimizaciones para \mbox{FPGA}, aunque no existen motivos por los que nuestras modificaciones no funcionasen con otras configuraciones.

\subsection{Microarquitectura}
El \textit{core} del SweRV-EL2 está segmentado en 4 etapas y se emite una sola instrucción por ciclo, por lo que es \textit{escalar}. Está diseñado para conseguir un \textit{IPC} (Instrucciones por Ciclo) cercano a $1.0$, logrando $0.95\ IPC$ y $0.98\ IPC$ en las pruebas \textit{Coremark} y \textit{Dhrystone}.

La figura \ref{fig:swerv_complex} presenta un esquema de las etapas del core. En primer lugar, se ejecuta la etapa de \textit{Fetch}, en la que se obtiene la instrucción de la memoria y se guarda en unos registros. A continuación, se hace la decodificación (\textit{Decode}) de la instrucción, obteniendo la información relevante: si es una instrucción de salto, una operación aritmético-lógica, los registros fuentes y destino, el inmediato codificado, etc. Después, dependiendo del tipo de instrucción puede ejecutarse uno u otro \textit{pipe} (fase \textit{Execute}), para continuar salvando los resultados durante la fase de \textit{Commit}.

Las instrucciones de multiplicación pasarán por el \textit{multiply pipe} que hace uso del multiplicador, con una latencia de 1 ciclo. Del mismo modo, las instrucciones de división pasarán por un \textit{pipe} no segmentado que utiliza el divisor, con una latencia de 34 ciclos. El resto de instrucciones usarán el \textit{pipe} principal ---llamado \textit{I0}---, excepto las de tipo \textit{load} y \textit{store}, que también cuentan con un \textit{pipe} dedicado. 
En este proyecto hemos incluido una NoC que solo será usada por los \textit{pipes} de multiplicación y división.
% En la figura \ref{fig:swerv_complex} podemos encontrar un diagrama con las distintas fases de una instrucción.

\begin{figure}[h]
    \centering
    \includegraphics[width=.7\textwidth]{images/diagrams/swerv_architecture.drawio.png}
    \caption[Fases de una instrucción ejecutada en el SweRV-EL2.]{Fases de una instrucción ejecutada en el SweRV-EL2. Señalado en naranja las fases que hacen uso de la NoC. Extraído de \citetitle{SweRVRoadmap}~\cite{SweRVRoadmap}.}
    \label{fig:swerv_complex}
\end{figure}

\subsection{Diseño RTL}

En cuanto al diseño RTL, programado en SystemVerilog, el \textit{top-module} del procesador es \textit{el2\_swerv\_wrapper}, que instancia y conecta la memoria (\textit{el2\_mem}) y la interfaz DMI (\textit{dmi\_wrapper}) con el \textit{core} (\textit{el2\_swerv}).

En la figura \ref{fig:swerv_fu} se muestra el diagrama de bloques del diseño simplificado y las unidades funcionales con las que cuenta el \textit{core}, cada una con las siguientes responsabilidades:

\begin{itemize}[noitemsep]
    \item [\textbf{ifu}] \textit{Instruction Fetch Unit}. Unidad funcional encargada de la fase \textit{fetch} de una instrucción. Se conecta con la memoria y obtiene la instrucción especificada por el contador del programa.
    \item [\textbf{dec}] Se encarga de la decodificación de las instrucciones obtenidas por la \textit{ifu}, obteniendo los operandos para la \textit{exu} y manejando las señales de control de esta.
    \item [\textbf{exu}] \textit{Execution Unit}. Esta unidad funcional se encarga de ejecutar la instrucción, por lo que instancia distintos submódulos para los distintos \textit{pipes} de ejecución. Esta es la unidad funcional a la que se añadirá la NoC.
    \item [\textbf{lsu}] \textit{Load-Store Unit}. Se encarga de ejecutar las instrucciones de \textit{load} y \textit{store}, accediendo a la memoria.
\end{itemize}

En este proyecto añadiremos una NoC para conectar los submódulos de la EXU, que se explican más en profundidad en la sección \hyperref[subsec:exu_mods]{\ref{subsec:exu_mods}~\nameref{subsec:exu_mods}}.

\begin{figure}[h]
    \centering
    \includegraphics{images/schematics/swerv_blocks.drawio.pdf}
    % \missingfigure{Aquí pondré una figura con las distintas unidades funcionales del SweRV, y la NoC dibujada dentro de la ALU (sin mucho detalle)}
    \caption{Diagrama de bloques simplificado del diseño RTL del SweRV-EL2, mostrando las unidades funcionales del \textit{core}. En naranja se muestra la ubicación de la NoC añadida.}
    \label{fig:swerv_fu}
\end{figure}

