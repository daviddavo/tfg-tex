\chapter{Conclusiones}

En el proyecto se ha creado una NoC funcional y se ha integrado en un procesador \mbox{RISC-V} \mbox{SweRV-EL2} para comunicar los módulos de la unidad de ejecución. En la realización del trabajo se han cumplido los siguientes objetivos:

\begin{enumerate}[noitemsep]
    \item He llevado a cabo un estudio en profundidad sobre las redes de interconexión en procesadores y sus distintas características.
    \item He aprendido a (i) describir diseño hardware en SystemVerilog y realizar \textit{testbenches} automatizados, (ii) sintetizar en Xilinx Vivado y (iii) simular y depurar usando \mbox{QuestaSim}. Además del manejo del lenguaje de scripting TCL, usado por dichas herramientas.
    \item He creado el diseño RTL sintetizable de una NoC, comprobando su funcionamiento siguiendo la metodología \textit{constrained random tests}.
    \item Me he familiarizado con la arquitectura RISC-V y, en concreto, con el diseño RTL y las herramientas del procesador SweRV-EL2.
    \item He aprendido el manejo del conjunto de herramientas incluidas en la distribución para la configuración, enlazado, compilación y simulación de tests de alto nivel con el procesador SweRV-EL2.
    \item He modificado el RTL del procesador SweRV-EL2 para incluir en su unidad de ejecución la NoC desarrollada en el proyecto ---usada por el multiplicador y el divisor---, logrando pasar las distintas pruebas incluidas con el diseño.
    \item He sintetizado el diseño y obtenido una estimación de los recursos hardware necesarios para su implementación sobre una FPGA Xilinx Virtex-7 XC7VX485T-2FFG1761C, analizando el impacto de la NoC.
\end{enumerate}

% En conclusión, se ha conseguido modificar un procesador de arquitectura RISC-V añadiendo a su unidad de ejecución una NoC sintetizable, verificando su correcto funcionamiento.
En conclusión, el nuevo diseño del SweRV-EL2 incluye una NoC en su unidad de ejecución, con un sobrecoste en recursos hardware del 9\% en LUTs y 4\% en FFs con respecto al diseño original. 
No obstante, la adición de la NoC permite usar técnicas de reconfiguración dinámica parcial cuando se detecten daños permanentes sobre parte del circuito, aplicando técnicas de tolerancia a fallos que hacen posible usar este diseño en entornos especialmente agresivos con los dispositivos electrónicos.

\begin{otherlanguage}{english}
\addtocounter{chapter}{-1}
\chaptertype{E}
\chapter{Conclusions}
In this project, I have created a working NoC and integrated it in the \mbox{RISC-V} \mbox{SweRV-EL2} processor to connect the submodules of its execution unit. 
% no me gusta "carry out"
% While carrying out this project, the following objectives have been achieved:
The following objectives have been fulfilled:

\begin{enumerate}[noitemsep]
    \item I have studied interconnection networks and their characteristics in-depth.
    % SI EL PLURAL DE BENCH ES BENCHES, EL PLURAL DE TESTBENCH ES TESTBENCHES
    \item I have learnt (i) to design hardware using SystemVerilog and automatized testbenches, (ii) to synthesize with Xilinx Vivado, and (iii) to simulate and debug using \textit{QuestaSim}. I also have acquired the handling of the TCL scripting language used by these tools.
    \item I have created a synthesizable design of an NoC and checked its behaviour using the constrained random tests methodology.
    \item I have become familiar with the RISC-V architecture and, in particular, with the RTL design and tools of the SweRV-EL2 processor.
    \item I have learnt to utilize the toolset included in the repository to configure, link, compile, and run software tests on the SweRV-EL2 processor. 
    \item I have modified the RTL design of the SweRV-EL2 to include an NoC inside its execution unit ---used by its multiplier and divisor---, succeeding in passing the various tests included with the design.
    \item I have synthetized the design and obtained an estimation of the hardware resources needed to implement it on a Xilinx Virtex-7 FPGA (XC7VX485T-2FFG1761C), analysing the impact of the NoC.
\end{enumerate}

% In conclusion, a RISC-V architecture processor has been modified by adding a NoC to its execution unit, verifying its correct operation and that it remains being synthesizable.
In conclusion, the new SweRV-EL2 design includes an NoC within its execution unit, with an additional cost in hardware resources of 9\% in LUTs and 4\% in FFs compared to the original design.
Nevertheless, adding the NoC allows using dynamic partial reconfiguration methods to recover processor functionality when permanent damage in the FPGA is detected, improving the design fault-tolerant capabilities and allowing its use in especially harsh environments for electronic devices.
\end{otherlanguage}