\addchap*{Agradecimientos}

En primer lugar quiero agradecer a mi novia, Leidy, por estar conmigo durante más de 7 años queriéndome, apoyándome y ayudándome y, sobre todo, por recordarme que de vez en cuando hay que descansar un poquito y despejarse.

También quiero dar las gracias a mi familia; en particular a mis padres, Anselmo y Gloria, por aguantarme mientras duraba el proyecto y desde siempre, y a mi tía, Toñi, por los consejos de estilo y ayuda con la memoria.

Este trabajo tampoco habría sido posible sin mis tutores del TFG, Óscar y Juan, que han estado semana por semana siguiéndome y aportándome ayuda cuando más atascado me encontraba. También a muchos de mis antiguos profesores, sobre todo a aquellos \textit{de letras} que confiaron en mí cuando nadie más lo hizo y me hicieron cambiar.

A la gente de la universidad que no puede estarse quieta: FDIst, OTEA y ASCII que, a pesar del poco tiempo libre con el que cuentan sus profesores y estudiantes, deciden sacar un poco para dedicarlo a hacer cosas chulas (y, a veces, dudosamente legales).

También a los compañeros y amigos que hice por el camino. Aunque ya no nos veamos tanto sé que seguís ahí y merecéis mi gratitud: Zero, Leila y Victor, Pont, Ela y Null, Pascal, Marina, Daster, Yago y muchos más... siempre me acordaré de todos vosotros. Igual a mis amigos del pueblo: Mario, Sandra, Paco, Rubén, Aritz, Hugo, Carlos... sé que cuando empecé a estudiar cada vez más, y más lejos, fuimos perdiendo el contacto, pero tampoco olvidaré esos años de juegos online nocturnos en los que comenzaron mis andaduras de sysadmin.

No sin olvidar a los chicos y chicas de MEL, iniciativa del pueblo con la que crecí y me eduqué y con la que se me presentaron oportunidades increíbles. En especial a mis educadores, Esther y Jorge, que supieron ver en mí algo que el resto no quiso. Sin ellos, desde luego no sería quien soy hoy.

Finalmente, me gustaría agradecer a todos aquellos gigantes sobre los que se apoya nuestro trabajo, a todos los hackers, piratas y magos que se quedan hasta tarde, a los que dieron el alma a una nueva máquina, la máquina de los sueños, y crearon imperios, catedrales y bazares. Gracias a todos ellos que, a lo largo de décadas de historia han permitido que este trabajo, y muchos otros, sean posibles.